\begin{notes}
    Story of Commutative Algebra. (Eisenbud 1)
\end{notes}

\begin{enumerate}
    \item
          Gauss proved that $\mathbb{Z}[i]$ is a UFD and used this
          on 1928 paper on biquadratic residues to prove results about
          ordinary numbers.
    \item
          Euler, Gauss, Dirichlet, and Kummer used this idea for $\mathbb{Z}
              [\zeta]$, with $\zeta$ a $nth$ root of unity,
          to prove some special cases of Fermat's last theorem.
    \item
          $\mathbb{Z}[\zeta]$ doesn't always have unique factorization. (first example
          $n$=23) The search for generalization of unique factorization birthed Dedekind's
          idea of ideals of a ring.
    \item
          The search culminated in two major theories:
          Dedekind's unique factorization of ideals into prime ideals (Dedekind
          domains); and Kronecker's theory of polynomial rings and Lasker's theory
          of primary decomposition in them.
    \item
          Dedekind represented an element $r\in R$ by the ideal $(r)$ and found conditions
          under which a ring has unique factorization of ideals into prime ideals. (the ring
          of all integers in any number field)
    \item
          Kronecker put the notion of adjoining the root of polynomial to a field $k$ on firm footing
          by introducing the ring $k[x]$, with the desired ring being $k[x]/f(x)$. There is no way to
          factorize ideals in a polynomial rings but Lasker later showed how to generalize unique
          factorization into "primary decomposition".

\end{enumerate}