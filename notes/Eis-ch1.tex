\begin{notes}
      Story of Commutative Algebra. (Eisenbud 1)
\end{notes}

\begin{enumerate}
      \item
            Gauss proved that $\mathbb{Z}[i]$ is a UFD and used this
            on 1928 paper on biquadratic residues to prove results about
            ordinary numbers.
      \item
            Euler, Gauss, Dirichlet, and Kummer, then, used this idea for $\mathbb{Z}
                  [\zeta]$, with $\zeta$ a $nth$ root of unity,
            to prove some special cases of Fermat's last theorem. The idea required factorization
            of $x^n+y^n$ over $\mathbb{Z}[\zeta]$.
      \item
            $\mathbb{Z}[\zeta]$ doesn't always have unique factorization. (first example
            $n$=23) The search for generalization of unique factorization birthed Dedekind's
            idea of ideals of a ring.
      \item
            This search culminated in two major theories:
            Dedekind's unique factorization of ideals into prime ideals (Dedekind
            domains); and Kronecker's theory of polynomial rings and Lasker's theory
            of primary decomposition in them.
            \begin{enumerate}
                  \item[a.]
                        Dedekind represented an element $r\in R$ by the ideal $(r)$ and found conditions
                        under which a ring has unique factorization of ideals into prime ideals. (the ring
                        of all integers in any number field)
                  \item[b.]
                        Kronecker put the notion of adjoining the root of polynomial to a field $k$ on firm footing
                        by introducing the ring $k[x]$, with the desired ring being $k[x]/f(x)$. There is no way to
                        factorize ideals in a polynomial rings but Lasker later showed how to generalize unique
                        factorization into "primary decomposition".
            \end{enumerate}
      \item
            Around 1860, Abel, Jacobi and Riemann made entirely new view of algebraic curves possible.
            From 1875-1882, Kronecker, Wierstrass, Dedekind, and Weber discovered that algebraic techniques
            that were developed to handle number fields could be applied to geometrically defined fields, thus
            pioneering the "arithmetic approach to function theory."
      \item
            After Pl$\ddot{\text{u}}$cker introduced projective coordinates around 1830, people were
            interested in the geometric invariant properties under certain classes of transformations. One way to
            express such an invariant property leads to finding an associative number which is invariant under
            choice of coordinates.
            \begin{enumerate}
                  \item[a.]
                        Mathematicians realized that the invariance under choice of coordinates was the invariance
                        under an action of a group (GL$_n(k)$ or SL$_n(k)$).
                  \item[b.]

            \end{enumerate}
\end{enumerate}