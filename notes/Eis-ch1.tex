\begin{notes}
      Story of Commutative Algebra. (Eisenbud 1)
\end{notes}
\hrule

\begin{enumerate}
      \item
            Gauss proved that $\mathbb{Z}[i]$ is a UFD and used this on 1928 paper on biquadratic residues to prove
            results about ordinary numbers.
      \item
            Euler, Gauss, Dirichlet, and Kummer, then, used this idea for $\mathbb{Z}
                  [\zeta]$, with $\zeta$ a $nth$ root of unity,
            to prove some special cases of Fermat's last theorem. The idea required factorization
            of $x^n+y^n$ over $\mathbb{Z}[\zeta]$.
      \item
            $\mathbb{Z}[\zeta]$ doesn't always have unique factorization. (first example
            $n$=23) The search for generalization of unique factorization birthed Dedekind's
            idea of ideals of a ring.
      \item
            This search culminated in two major theories:
            Dedekind's unique factorization of ideals into prime ideals (Dedekind
            domains); and Kronecker's theory of polynomial rings and Lasker's theory
            of primary decomposition in them.
            \begin{enumerate}
                  \item[a.]
                        Dedekind represented an element $r\in R$ by the ideal $(r)$ and found conditions
                        under which a ring has unique factorization of ideals into prime ideals. (the ring
                        of all integers in any number field)
                  \item[b.]
                        Kronecker put the notion of adjoining the root of polynomial to a field $k$ on firm footing
                        by introducing the ring $k[x]$, with the desired ring being $k[x]/f(x)$. There is no way to
                        factorize ideals in a polynomial rings but Lasker later showed how to generalize unique
                        factorization into "primary decomposition".
            \end{enumerate}
      \item
            $[$Algebraic Curves and Function Theory]
            Around 1860, Abel, Jacobi and Riemann made entirely new view of algebraic curves possible.
            From 1875-1882, Kronecker, Wierstrass, Dedekind, and Weber discovered that algebraic techniques
            that were developed to handle number fields could be applied to geometrically defined fields, thus
            pioneering the "arithmetic approach to function theory."
      \item
            $[$Invariant Theory]
                  After Pl$\ddot{\text{u}}$cker introduced projective coordinates around 1830, people were
            interested in the geometric invariant properties under certain classes of transformations. One way to
            express such an invariant property leads to finding an associative number which is invariant under
            choice of coordinates.
            \begin{enumerate}
                  \item[a.]
                        Mathematicians realized that the invariance under choice of coordinates was the invariance
                        under an action of a group (GL$_n(k)$ or SL$_n(k)$).
                  \item[b.]
                        The general problem, then, became finding the set of invariants (a subalgebra of $S$) $S^G$ under
                        "nice" action of a group $G$ of automorphisms of polynomial ring $S=k[x_1,\ldots,x_n]$.
                        The \textbf{fundamental problem of invariant theory} was the problem of existence of
                        finite systems of generators of $S^G$.
                  \item[c.]
                        In a series of papers from 1888 to 1893, Hilbert showed that the ring of invariants is finitely
                        generated in a wide range of cases. Aside from this, Hilbert proved four major results (the basis
                        theorem, the Nullstellensatz, the polynomial nature of the Hilbert function and the syzygy theorem),
                        all of which played enormous role in development of commutative algebra.
            \end{enumerate}
      \item
            A graded ring is a ring $R$ together with the direct sum decomposition
            $$R=R_0\oplus R_1 \oplus R_2\oplus\cdots \text{ as abelian groups},$$
            such that $R_iR_j \subset R_{i+j}$ for $i,j\geq 0$.
            \begin{mybox}{The Basis Theorem \& Finite Generation of Invariants}
                  \begin{thm}[Hilbert Basis Theorem]
    \label{hilbertbasis}
    If a ring $R$ is Noetherian, then the polynomial ring $R[x]$ is Noetherian.
\end{thm}
\begin{cor}
    \label{hilbertbasiscor1}
    Any homomorphic image of a Noetherian ring is Noetherian. Furthermore, if $R_0$ is
    a Noetherian ring, and $R$ is a finitely generated algebra over $R_0$, then $R$ is
    Noetherian.
\end{cor}
\begin{prop}
    \label{hilbertbasisprop1}
    If $R$ is a Noetherian ring and $M$ is a finitely generated $R$-module, then M is
    Noetherian.
\end{prop}
                  \begin{cor}
    \label{invariantsfiniteness}
    Let $k$ be a field, $S=k[x_1,\ldots,x_r]$ be a polynomial ring graded by degree, and $R$ a $k$-subalgebra
    of $S$. If $R$ is a summand of $S$, in the sense that there is a map of $R$-modules $\varphi:S\to R$ that
    preserves degrees and takes each element of $R$ to itself, then $R$ is a finitely generated $k$-algebra.
\end{cor}
                  Hilbert's finiteness result follows by taking $R=S^G$ above.
            \end{mybox}

      \item
            Given a subset $I\subset k[x_1,\ldots,x_n]$, the algebraic subset of $k^n$
            $$Z(I)=\{(a_1,\ldots,a_n)\in k^n | f(a_1,\ldots,a_n)=0\text{ for all }f\in I\}$$
            is called \textbf{algebraic variety} if it is \textbf{irreducible} (not the union of two smaller
            algebraic subsets). Taking algebraic sets as closed sets, we obtain \textbf{Zariski topology} on $k^n$.
            Given a set $X\subset k^n$, $$I(X) =\{f\in k[x_1,\ldots,x_n]|f(a_1,\ldots,a_n)=0\text{ for all }
                  (a_1,\ldots,a_n)\in X\}$$ is an ideal of $k[x_1,\ldots,x_n]$. Identifying polynomial functions that
            agree at all points of a set $X$, we get the \textbf{coordinate ring} $A(X)=k[x_1,\ldots,x_n]/I(X)$ of X
            which is reduced. The ideals $I(X)$ are all radical ideals, but not all radical ideals can appear as $I(X)$
            (but in an algebraically closed field, all radical ideals appear as such).

            \begin{mybox}{The Nullstellensatz}
                  \begin{thm}[Nullstellensatz]
    \label{Nullstellensatz}
    Let $k$ be an algebraically closed field. If $I\subset k[x_1,\ldots,x_n]$ is an ideal, then
    $$I(Z(I))=\sqrt{I}.$$
    Thus the correspondences $I\mapsto Z(I)$ and $X\mapsto I(X)$ induces a bijection between the collection of algebraic
    subsets of $\mathbf{A}^n_k$ and radical ideals of $k[x_1,\ldots,x_n]$.
\end{thm}

\begin{cor}
    A system of polynomial equations $\{f_1=0,\ldots, f_m=0\}$ over an algebraically closed field $k$ has no solutions
    in $k^n$ iff $1$ can be expressed as a linear combination $1=\sum{p_if_i}$ with polynomial coefficients $p_i$.
\end{cor}

\begin{cor}
    If $k$ is an algebraically closed field and $A$ is a $k$-algebra, then $A=A(X)$ for some algebraic set $X$ iff $A$
    is reduced and finitely generated as $k$-algebra.
\end{cor}

\begin{cor}
    Let $k$ be an algebraically closed field and let $X\subset \mathbf{A}^n$ be an algebraic set. Every maximal ideal
    of $A(X)$ is of the form $\maxm_p:=(x_1-a_1,\ldots,x_n-a_n)/I(X)$ for some $p=(a_1,\ldots,a_n)\in X$.
\end{cor}

\begin{cor}
    The category of affine algebraic sets and morphisms (over an algebraically closed field $k$) is equivalent to the
    affine $k$-algebras with the arrows reversed.
\end{cor}
            \end{mybox}

      \item
\end{enumerate}