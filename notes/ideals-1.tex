\begin{notes}
      Basic properties of ideals. (Matsumura 1)
\end{notes}

\begin{enumerate}
      \item
            In a surjective ring homomorphism $f:A \to A/I$, the ideals $J$
            of $A/I$ and the ideals $f^{-1}(J)$ of $A$ are in one-to-one
            correspondence. (lattice isomorphism theorem) When we need to think about ideals of $A$
            containing $I$, we can work on $A/I$: if $I'$ is any ideal of
            $A$ then $f(I')$ is an ideal of $A/I$ with $f^{-1}(f(I'))=I+I'$,
            and $f(I') = (I+I')/I$.
      \item
            For $a \in A$, $(a)=(1)$ iff $a$ has an inverse in A.
            If $a$ is a unit and $x$ is nilpotent, then $a+x$ is a unit.
      \item
            Using Zorn's lemma on a set of proper ideals (ordered by inclusion)
            containing some ideal $I$, we see that
            there exists at least one maximal ideal $M$ containing $I$.
            $A/M$ is a field.
      \item
            A proper ideal $P$ is prime if $x,y \notin P \implies xy \notin
                  P$. $A/P$ is an integral domain.
      \item
            If $I$ is an ideal disjoint from a multiplicative set $S$, then $A-S$
            has a maximal ideal containing I which is prime. (prove...)
      \item
            If $I$ is an ideal, then the radical of $I$
            $$ \sqrt{I}=\{a\in A : a^n \in I \text{ for some } n>0\}$$
            is also an ideal. If $P$ is a prime ideal containing $I$ then $\sqrt{I}
                  \subset P$. Furthermore, if $x \notin \sqrt{I}$, then we can find
            a prime ideal containing $\sqrt{I}$ but not $x$. And hence,
            $$\sqrt{I}=\bigcap_{P \supset I}{P}$$
      \item
            The intersection of all prime ideals is nilradical nil$(A)$ and the
            intersection of all maximal ideals is Jacobson radical rad$(A)$.
            $x\in \text{rad}(A) \text{ iff } 1+Ax$ consists entirely of units of A.
      \item\label{idealsinduction}
            $II' \subset I\cap I'$. If $I+I'=(1)$, then $II' = I\cap I'$. Also,
            if $I+I'=(1)$ and $I+I''=(1)$, then $I+I'I'' = (1)$. Hence, for ideals
            $I_i$ which are coprime in pairs
            $$I_1I_2...I_n=I_1\cap I_2 \cap ... \cap I_n.$$
      \item
            If $I+I' = (1)$, then $A/II'\simeq A/I \times A/I'$. This can be
            extended with $n$ ideals like in \ref{idealsinduction}.
      \item
            A maximal ideal $\mathfrak{m}$ of $A$ corresponds with the maximal
            ideal $\mathbf{m}=\mathfrak{m}B + (X_1,...,X_n)$ of the ring
            $B=A[[X_1,...,X_n]]$. Here, $\mathbf{m}\cap A = \mathfrak{m}$. These
            properties are not necessarily true in case of polynomial rings.
      \item
            $a\in A$ is called irreducible element if $a$ is not a unit of A
            and \begin{center}
                  $a=bc\implies b $ or $c$ is a unit of A.
            \end{center}
            $a$ is irreducible iff $aA$ is maximal among proper principal ideals
            and $a$ is prime if $aA$ is a prime ideal.
      \item
            $A=\mathbb{Z}[\sqrt{-5}]=\mathbb{Z}[X]/(X^2+5)$; then setting
            $k=\mathbb{Z}/2\mathbb{Z}$ we have
            \begin{center}
                  $A/2A=\mathbb{Z}[X]/(2,X^2+5)=k[X]/(X^2-1)=k[X]/(X-1)^2$.
            \end{center}
            Then $P=(2,1-\sqrt{-5})$ is a maximal ideal of $A$ containing $2$.
\end{enumerate}